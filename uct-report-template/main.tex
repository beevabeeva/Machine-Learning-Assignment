\documentclass[12pt]{book}
\usepackage[english]{babel}
\usepackage{natbib}
\usepackage{url}
\usepackage[utf8x]{inputenc}
\usepackage{amsmath}
\usepackage{graphicx}
\graphicspath{{images/}}
\usepackage{parskip}
\usepackage{fancyhdr}
\usepackage{vmargin}
\usepackage{array}
\usepackage{tabularx}
\setmarginsrb{3 cm}{2.5 cm}{3 cm}{2.5 cm}{1 cm}{1.5 cm}{1 cm}{1.5 cm}

\title{Assignment}								% Title
\author{A.E Bank \newline N. Blundell}
\date{\today}											% Date

\makeatletter
\let\thetitle\@title
\let\theauthor\@author
\let\thedate\@date
\makeatother

% \pagestyle{fancy}
\fancyhf{}
\cfoot{\thepage}

\begin{document}

%%%%%%%%%%%%%%%%%%%%%%%%%%%%%%%%%%%%%%%%%%%%%%%%%%%%%%%%%%%%%%%%%%%%%%%%%%%%%%%%%%%%%%%%%

\begin{titlepage}
	\centering
    \vspace*{0.5 cm}
    \includegraphics[scale = 0.75]{wits-logo.jpg}\\[1.0 cm]	% University Logo
    \textsc{\LARGE University of the Witwatersrand}\\[2.0 cm]	% University Name
	\textsc{\Large COMS3007}\\[0.5 cm]				% Course Code
	\textsc{\large Machine Learning}\\[0.5 cm]				% Course Name
	\rule{\linewidth}{0.2 mm} \\[0.4 cm]
	{ \huge \bfseries \thetitle}\\
	\rule{\linewidth}{0.2 mm} \\[1.5 cm]

	\begin{minipage}{0.4\textwidth}
		\begin{flushleft} \large
			\emph{Authors:}\\
			\theauthor
			\end{flushleft}
			\end{minipage}~
			\begin{minipage}{0.4\textwidth}
			\begin{flushright} \large
			\emph{Student Numbers:} \\
				604124 \\
				1132328									% Your Student Number
		\end{flushright}
	\end{minipage}\\[2 cm]

	{\large \thedate}\\[2 cm]

	\vfill

\end{titlepage}

%%%%%%%%%%%%%%%%%%%%%%%%%%%%%%%%%%%%%%%%%%%%%%%%%%%%%%%%%%%%%%%%%%%%%%%%%%%%%%%%%%%%%%%%%

\tableofcontents
\pagebreak

%%%%%%%%%%%%%%%%%%%%%%%%%%%%%%%%%%%%%%%%%%%%%%%%%%%%%%%%%%%%%%%%%%%%%%%%%%%%%%%%%%%%%%%%%
\chapter{Introduction}
This is where we put an introduction.
In addition to the assignment requirements, we explore and discuss methods of analysing categorical data.

\chapter{Background}
\section{Data Set}
The data set used for this assignment is the Mushroom Data Set.\cite{dataset}
It is based on records from The Audubon Society Field Guide to North American Mushrooms.
The samples are from 23 species of the Agaricus and Lepiota Family.
Each sample has 22 attributes and is classified as either poisonous or edible.
The attributes are as follows:
\begin{center}
\begin{tabularx}{\linewidth}{ |l|X| }
 \hline
 1. cap-shape & bell=b, conical=c, convex=x, flat=f, knobbed=k, sunken=s \\
 \hline
 2. cap-surface & fibrous=f, grooves=g, scaly=y, smooth=s \\
 \hline
 3. cap-colour & brown=n, buff=b, cinnamon=c, gray=g, green=r, pink=p, purple=u, red=e, white=w, yellow=y \\ \hline 4. bruises? & bruises=t, no=f \\
 \hline
 5. odor & almond=a, anise=l, creosote=c, fishy=y, foul=f, musty=m, none=n, pungent=p, spicy=s \\ \hline 6. gill-attachment & attached=a, descending=d, free=f, notched=n \\
 \hline
 7. gill-spacing & close=c, crowded=w, distant=d \\ \hline 8. gill-size & broad=b, narrow=n \\
 \hline
 9. gill-colour & black=k, brown=n,buff=b, chocolate=h, gray=g, green=r,orange=o, pink=p, purple=u, red=e, white=w, yellow=y \\
 \hline
 10. stalk-shape & enlarging=e, tapering=t \\ \hline 11. stalk-root & bulbous=b, club=c, cup=u, equal=e, rhizomorphs=z, rooted=r, missing=? \\
\hline
 12. stalk-surface-above-ring & fibrous=f, scaly=y, silky=k, smooth=s \\ \hline 13. stalk-surface-below-ring & fibrous=f, scaly=y, silky=k, smooth=s \\
\hline
 14. stalk-colour-above-ring & brown=n, buff=b, cinnamon=c, gray=g, orange=o, pink=p, red=e, white=w, yellow=y \\
\hline
 15. stalk-colour-below-ring & brown=n, buff=b, cinnamon=c, gray=g, orange=o, pink=p, red=e, white=w, yellow=y \\
\hline
 16. veil-type & partial=p, universal=u \\ \hline 17. veil-colour & brown=n, orange=o, white=w, yellow=y \\ \hline 18. ring-number & none=n, one=o, two=t \\
 \hline
 19. ring-type & cobwebby=c, evanescent=e, flaring=f, large=l, none=n, pendant=p, sheathing=s, z one=z \\
 \hline
 20. spore-print-colour & black=k, brown=n, buff=b, chocolate=h, green=r, orange=o, purple=u, white=w, yellow=y \\
 \hline
 21. population & abundant=a, clustered=c, numerous=n, scattered=s, several=v, solitary=y \\ \hline 22. habitat & grasses=g, leaves=l, meadows=m, paths=p, urban=u, waste=w, woods=d \\
\hline
\end{tabularx}
\end{center}




The Field Guide apparently (it is not free) states that there is no simple rule for determining whether a mushroom from one of these families is edible or not.

\section{About this design}
This is a simple report template with the UCT logo. Feel free to use/modify it to suit your needs. Variables that need to be altered have been commented to make modifications easier. For example if you need to change the university logo, look for the comment \texttt{\% University Logo} in this file and then make appropriate modifications in that line.

A Table of Contents and a bibliography have also been implemented. To add entries to your bibliography, simply edit \texttt{biblist.bib} in the root folder and then use the \texttt{\textbackslash cite\{\ldots\}} command in \texttt{main.tex} \cite{bibtex}. The Table of Contents will be updated automatically.

I hope that you find this template both visually appealing and useful.

\hspace{1 cm}--- Lin

\newpage
\bibliographystyle{plain}
\bibliography{biblist}
\end{document}
